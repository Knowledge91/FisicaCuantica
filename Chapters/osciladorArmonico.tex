\chapter{Oscilador Armónico Cuántico}

\section{Problemas}

\begin{enumerate}

\item Ecuentra las expressiones del los observables $x$ y $p$ en términos de los
operadores $a$ y $a^\dagger$ que permiten escribir l'hamiltoniano armónico
unidimensional como $H = \hbar \omega (a^\dagger a + 1/2)$. Conviene que
utilizas argumentos d'hermitinidad y dimensional.	\\\\
\textbf{Solucion:} \\
Los operadores escalera estan definida por
\begin{align*}
	a \,=\, \sqrt{\frac{m\omega}{2 \hbar}} \left( \hat x + \frac{i}{m \omega} \hat p
\right)	\\
	a^\dagger \,=\, \sqrt{\frac{m\omega}{2 \hbar}} \left( \hat x - \frac{i}{m \omega}
\hat p\right)
\end{align*}
Así añadiendo y sustraiendo los operadores escalar danos
\begin{align*}
	a + a^\dagger \,=\,a \sqrt{\frac{m \omega}{2 \hbar}} ( \hat x + \hat x) &
\Rightarrow \quad \hat x \,=\, \sqrt{\frac{\hbar }{2 m \omega}} (a + a^\dagger) \\
	a - a^\dagger \,=\,a \sqrt{\frac{m \omega}{2 \hbar}} \left(\frac{i}{m
\omega} \hat p + \frac{i}{m \omega} \hat p \right) a \Rightarrow \quad \hat p
\,=\, \sqrt{\frac{\hbar m \omega}{2}} (-i) (a - a^\dagger)
\end{align*}
Ahora vamos a comprobar la dimensionalidad. En general los unidades utilizados
para los operadores escalares estan
\begin{align*}
	m \,=\, kg \quad \omega \,=\, \sqrt{\frac{k}{m}} \,=\, \frac{1}{s} \quad h
\,=\, \frac{kg \cdot m^2}{s},
\end{align*}
porque $h = J\cdot s = N \cdot m \cdot s$ y $k = N / m = kg / s$. Así que la
comproba de $\hat x$
\begin{equation*}
	\hat x \,=\, \sqrt{\frac{h}{m \omega}} \,=\, \sqrt{\frac{kg \cdot m^2}{s} 
\cdot kg^{-1} \cdot s} \,=\, m
\end{equation*}
y de $\hat p$
\begin{equation*}
	\hat p \,=\, \sqrt{h m \omega} \,=\, \sqrt{\frac{kg \cdot m^2}{s} \cdot kg
\cdot s^{-1}} \,=\, \frac{kg \cdot m}{s}
\end{equation*}
donde hemos mirado solo términos importantes, estan hecho facilmente.

\item Utilza los operadores escalar $a$ y $a^\dagger$ para calcular los valores
esperados $\langle x \rangle_n$, $\langle p \rangle_n$, $\langle x^2 \rangle_n$,
$\langle p^2 \rangle_n$, $\langle K \rangle_n$, $\langle V \rangle_n$ y los
indeterminaciónes $\langle (\Delta x)^2 \rangle_n$, $\langle(\Delta p)^2\rangle_n$ y
$\langle(\Delta H)^2\rangle_n$ de el estado
estacionario $|n\rangle$ de l'oscilador armónico unidimensional. \\\\
\textbf{Solucion:}\\
Recuerdando que los vectores del estado estan orthogonales
\begin{equation*}
	\langle n | n' \rangle \,=\, \delta_{nn'}
\end{equation*}
nos podemos calcular $\langle \hat x \rangle_n$ y $\langle \hat p \rangle_n$
facilmente 
\begin{align*}
	\langle \hat x \rangle_n a= \langle n | \hat x | n \rangle =
\sqrt{\frac{\hbar}{2m\omega}} \langle n| a^\dagger + a | n \rangle 
	= \sqrt{\frac{\hbar}{2m\omega}} (\langle n | a^\dagger | \rangle +
\langle n | a | n \rangle ) \\
	a = \sqrt{\frac{\hbar}{2m\omega}} (\sqrt{n+1} \langle
n| n+1\rangle + \sqrt{n} \langle n|n-1\rangle) = 0,\\
	\langle \hat p \rangle_n a= \sqrt{\frac{\hbar}{2m\omega}} (-i) \langle n |
(a - a^\dagger) | n \rangle = 0.
\end{align*}
Utilizando el operador número $N = a^\dagger a$ y el commutador de los
operadores escalar
\begin{equation*}
	[a, a^\dagger] = aa^\dagger - a^\dagger a = 1 \quad \Rightarrow \qquad a
a^\dagger = a^\dagger a + 1 = \hat N + 1	
\end{equation*}
danos las valores esperados de $\langle \hat x^2 \rangle_n$ y $\langle \hat p^2
\rangle_n$
\begin{align*}
	\langle \hat x^2 \rangle_n a= \langle n | \hat x^2 | n \rangle =
\frac{\hbar}{2m\omega} (\langle n | \cancel{a^\dagger a^\dagger} +
\underbrace{a^\dagger a}_{\hat N} + \underbrace{aa^\dagger}_{\hat N + 1} +
\cancel{aa} | n\rangle  \\
	a= \frac{\hbar}{2m\omega} (\langle n | 2 \hat N + 1 | n \rangle =
\frac{\hbar}{2m\omega} (2n +1) = \frac{\hbar}{m\omega}
\left(n+\frac{1}{2}\right) \\
	\langle \hat p^2 \rangle_n a= -\frac{\hbar m \omega}{2} [-(2n +1] =
\frac{\hbar m \omega}{2} (2n + 1) = \hbar m \omega (n + \frac{1}{2}).	
\end{align*}
Los valores esperados kintetico $\langle K \rangle_n$ y potencial $\langle V
\rangle_n$ estan compuesto de los valores esperados calculado antes, por lo
tanto nos podemos escribir
\begin{align*}
	\langle K \rangle_n a= \frac{1}{2m} \langle \hat p^2 \rangle_n \frac{\hbar
\omega}{2} \left( n \frac{1}{2} \right) \\
	\langle V \rangle_n a=  \frac{1}{2} m \omega^2 \langle x^2 \rangle_n =
\frac{1}{2} \hbar \omega \left( n + \frac{1}{2} \right)
\end{align*}
Por esto $\langle H \rangle_n$  esta dado por
\begin{equation*}
	\langle H \rangle_n = \langle K \rangle_n + \langle V \rangle_n = \hbar
\omega \left( n + \frac{1}{2} \right).
\end{equation*}
Por los indeterminaviones nos recuerdamos de la relación de indeterminación
\begin{equation*}
	\Delta A \equiv A - \langle A \rangle \quad \Rightarrow \qquad \langle
(\Delta A)^2 \rangle = \langle A^2 \rangle - \langle A \rangle^2
\end{equation*}
Así los indeterminaciones estan dado por
\begin{align*}
	\langle (\Delta x)^2 \rangle_n a= \langle x^2 \rangle_n -
\underbrace{\langle x \rangle_n}_0 = \frac{\hbar}{m\omega} \left( n +
\frac{1}{2}\right) \\
	\langle (\Delta p)^2 \rangle_n a= \langle p^2 \rangle_n -
\underbrace{\langle p \rangle_n}_0 = \hbar m \omega \left(n + \frac{1}{2}
\right) \\
	\langle (\Delta H)^2 \rangle_n = \langle H^2 \rangle_n - \langle H
\rangle_n^2 = 0
\end{align*}
El ultimo realción esta verdad porque dando un Hamiltoniano armónico
\begin{equation*}
	H = \frac{1}{2m} p^2 + \frac{1}{2} m \omega^2 x^2 \quad \Rightarrow \qquad
H^2 = \frac{1}{4m^2} \underbrace{p^4}_{(a-a^\dagger)^4} + \frac{1}{4} m^2
\omega^2 \underbrace{x^4}_{(a+a^\dagger)^4}
\end{equation*}
así considerado solo $x^4$
\begin{equation*}
	\langle x^4 \rangle_n = \langle n |(a+a^\dagger)^4 | n \rangle = \langle n |(\cancel{aa} + aa^\dagger + a^\dagger a + \cancel{a^\dagger
a^\dagger})^2 | n \rangle = \langle n |(aa^\dagger + a^\dagger a)^2 | n \rangle
= \langle n | ( a a^\dagger + a^\dagger a) | n \rangle^2  
\end{equation*}
repitiendo el mismo processo por $p^4$ danos como resultado
\begin{equation*}
	\langle H^2 \rangle_n = \langle H \rangle_n^2
\end{equation*}
por lo tanto vemos que la indeterminacion del Hamiltoniano esta cero.

\item por el estado no estacionario inicial (y sencillo) $|\psi(t=0)\rangle =
\frac{1}{\sqrt{2}} |0\rangle + \frac{1}{\sqrt{2}} | 1 \rangle$ y cualquier
instant de tiempo t. \\\\
\textbf{Solucion:} \\
Primero tenemos que evaluar la evolución del tiempo del estado no esacionario
inicial. Por lo tanto deberiamos calcular la energía del oscillador armónico
del estado fundamental y primero estado
\begin{equation*}
	E_n = \hbar \omega (n + \frac{1}{2}), \quad E_0 = \frac{\hbar \omega}{2}
,\quad E_1 = \frac{3 \hbar \omega}{2}.
\end{equation*}
Así utilizando
\begin{equation*}
	|\alpha (t)\rangle = \sum_n \alpha_n e^{-i\frac{E_n}{\hbar} t} | n\rangle
\end{equation*}
danos la evolución temporal de $|\psi(t=0)\rangle$ 
\begin{equation*}
	|\psi(t)\rangle = \frac{1}{\sqrt{2}} e^{-\frac{i\omega}{2}}|0\rangle
+ \frac{1}{\sqrt{2}} e^{-\frac{3 i \omega t}{2}} |1 \rangle  = \frac{1}{\sqrt{2}}
e^{-\frac{i\omega t}{2}} \left( |0\rangle + e^{\frac{i \omega t}{2}} |
2\rangle \right)
\end{equation*}
Logicamente por los valores eseperados nos tenemos
\begin{align*}
	\langle \hat x \rangle_{\psi(t)} &= \langle \psi | \hat x | \psi \rangle \\
	&= \frac{1}{2} \cancel{e^{\frac{- i \omega t}{2}}e^{\frac{i \omega t}{2}}}
\frac{\hbar}{2m\omega} \left( \langle 0| + e^{i\omega t} \langle 1 | \right) ( a +
a^\dagger) \left( e^{i\omega t} | 1 \rangle + | 0 \rangle \right) \\
	&= \frac{1}{2} \sqrt{\frac{\hbar}{2m\omega}} \left( e^{-i\omega t}\langle 0
| a | 1 \rangle + e^{i\omega t} \langle 1 | a^\dagger | 0 \rangle \right) \\
	&= \frac{1}{2} \sqrt{\frac{\hbar}{2m\omega}} \left( e^{-i\omega t} + e^{i
\omega t} \right) \\
	&= \frac{1}{2} \sqrt{\frac{\hbar}{2m\omega}} \cos{\omega t}  
\end{align*}
y 
\begin{align*}
	\langle \hat p \rangle_{\psi(t)} &= \frac{1}{2} \sqrt{\frac{\hbar m
\omega}{2}} (-i) \left( \langle 0| + e^{i\omega t} \langle 1| \right) (a -
a^\dagger) \left( e^{-i\omega t} |1\rangle + |0\rangle \right) \\
	&= \frac{1}{2} \sqrt{\frac{\hbar m \omega}{2}} (-i) \left( e^{-i\omega
t}\langle 0 | a | 1 \rangle - e^{i\omega t} \langle 1| a^\dagger | 0 \rangle
\right) \\
	&= \sqrt{\frac{\hbar m \omega}{2}} \frac{1}{2i}\left( e^{-i\omega t} -
e^{i \omega t} \right) \\
	&= -\sqrt{\frac{\hbar m \omega}{2}} \sin{\omega t}  
\end{align*}

\item Dada la function d'onda del estado fundamental del oscilador armónico
unidimenional
\begin{equation*}
	\phi_0 (x) = \left(\frac{m\omega}{\pi \hbar}\right)^{\frac{1}{4}} e^{-\frac{1}{2}
\frac{m\omega}{\hbar} x^2}
\end{equation*}
ecuentre l'expresion de $\phi_2(x)$. \\\\
\textbf{Solucion:} \\
Nos podemos escribir el operador escalar $a^\dagger$ en la siguiente forma
\begin{equation*}
	a^\dagger = \frac{1}{\sqrt{2\hbar m\omega}}(m \omega \hat x - i \hat p) \quad
\text{con} \quad a^\dagger | n \rangle = \sqrt{n+1} | n+1 \rangle
\end{equation*}
Así
\begin{equation*}
	(a^\dagger )^2 | 0 \rangle = \sqrt{1} a^\dagger | 1 \rangle = \sqrt{2} | 2
\rangle \quad \Rightarrow \qquad | 2\rangle = \frac{1}{\sqrt{2}}(a^\dagger)^2 | 0 \rangle
\end{equation*}
y por lo tanto $\phi_2 (x)$ esta dado por
\begin{equation*}
	|2\rangle = \phi_2 = \frac{1}{\sqrt{2}} \phi_0.
\end{equation*}
Empezando con el operador escalar cuadrado
\begin{equation*}
	(a^\dagger )^2 = \frac{1}{2\hbar m \omega} (m \omega \hat x - i \hat p)^2 =
\frac{1}{2\hbar m \omega} \left(m \omega \hat x - \hbar \frac{\partial}{\partial
x}\right)^2 = \frac{1}{2\hbar m \omega} \left(m^2 \omega^2 x^2 + \hbar
\frac{\partial^2}{\partial x^2} - 2 m \omega \hbar x \frac{\partial}{\partial
x}\right)
\end{equation*}
donde hemos utilizado el operador momento $\hat p = - i \hbar \partial/\partial
x$. Antes de continuar queremos caclular la segunda derivada de 
\begin{equation*}
	\frac{\partial^2}{\partial x^2} \left(e^{-\frac{1}{2} \frac{m\omega}{\hbar}
x^2}\right) = \frac{\partial}{\partial x} \left(\frac{m \omega}{\hbar} x \right)
e^{-\frac{1}{2} \frac{m\omega}{\hbar} x^2} = \left(-\frac{m\omega}{\hbar} +
\frac{m^2\omega^2}{\hbar^2}x^2\right) e^{-\frac{1}{2}\frac{m\omega}{\hbar} x^2}
\end{equation*}
Por lo tanto $\phi_2$ esta dado por
\begin{align*}
	\phi_2 &= \frac{1}{\sqrt{2}} \frac{1}{2 \hbar m \omega}
\left(\frac{m\omega}{\pi\hbar}\right)^{\frac{1}{4}} \left(m^2 \omega^2 x^2 +
\hbar^2 \frac{\partial^2}{\partial x^2} - 2m\omega\hbar x
\frac{\partial}{\partial x} \right) e^{\frac{1}{2} \frac{m\omega}{\hbar}x^2} \\
	&= \frac{1}{\sqrt{2}} \frac{1}{2\hbar m\omega}
\left(\frac{m\omega}{\pi\hbar}\right)^{\frac{1}{4}} \left(m^2 \omega^2 x^2 + 2
m^2 \omega^2 x^2 - m \omega \hbar + m \omega \hbar + m^2 \omega^2 x^2 \right)
e^{\frac{1}{2} \frac{m \omega}{\hbar} x^2} \\
	&= \left(\frac{m\omega}{\pi\hbar}\right)^{\frac{1}{4}}
\left(-\frac{1}{\sqrt{8}} + \sqrt{2} \frac{m\omega}{\hbar} x^2 \right)
e^{\frac{1}{2} \frac{m\omega}{\hbar} x^2}
\end{align*}

\end{enumerate}

\chapter{Hamilton Mechanic}
In this chapter we want to further develop our theory introducing the
Hamitlonian.

In the last two chapters we developed the \textit{Lagrange Mechanic}, which was
valid in all coordinates systems and did not include \textit{forces of
constraints}\footnote{In the classical Newton Mechanics, one could only use
cartesian coordinates and had to use difficult forces of constraint to solve
mechanical problems.}. The \textit{Lagrangian formalism} was developed from the
\textit{d'Alembert principle} or \textit{Hamilton principle}, which subsituted
the axioms of Newton. Restricting ourself to \textit{holonom conservativ}
problems we have to deal with 2S differential equations of 2nd order for $q_1,
\cdots, q_S$ generalized coordinates. Implementing the need of 2S initial
conditions for the solution. 

Text missing
\section{Legendre Transformation}
Consider a function $f(x, y)$ dependend on two variables $x$ and $y$. Then the
differential of $df(x,y)$ is given by
\begin{equation}
  df(x,y) \,=\, \frac{\partial f}{\partial x} dx \frac{\partial f}{\partial y}
dy \,=\, udx + vdy,
\end{equation}
where we introduced
\begin{equation}
  u \,=\, \frac{\partial f}{\partial x} \quad \text{and} \quad v \,=\,
\frac{\partial f}{\partial y}.
\end{equation}
Now we want to find a function $g(x,v)$, which is dependen on $x$ and $v$
rather than on $x$ and $y$ (as in $f(x,v)$). So we want to change the
dependency
\begin{equation}
  y \,\to\, v 
\end{equation}
The differential of our desired function $g(x, v)$ is therefore given as
\begin{equation}
  dg(x,v) \,=\, \frac{\partial g}{\partial x} dx + \frac{\partial g}{\partial
v} dv.
\end{equation}
Thus regarding the differential function of our input function $df(x,y)$ once
again can rewrite it to 
\begin{equation}
  df(x, y) \,=\, udx + vdy \,=\, udx + d(vy) - ydv \quad \Rightarrow \quad d(vy
 - f) \,=\, - udx + ydv.
\end{equation}
Comparing the the differentials $d(vy -f)$ with $dg$ we see that they
have identical dependencies, thus
\begin{equation}
  dg \,=\, d(vy - f) \quad \text{with} u \,\equiv\, - \frac{\partial
g}{\partial x} \quad \text{and} \quad v \,\equiv\, \frac{\partial g}{\partial y}  
\end{equation}
Therefore we can define the \textit{Legendre transformed} as
\begin{equation}
  g(x, v) \,=\, vy - f \,=\, \frac{\partial f}{\partial y} y - f
\end{equation}

\section{The Hamilton Function and Hamitlon's Equations}
Using the Legendre transformation we can now transform
\begin{equation}
  L(\vec q, \dot{\vec q}, t) \,\to\, H(\vec q, \vec p, t).
\end{equation}
Remember that
\begin{equation}
  g(x, v) \,=\, \frac{\partial g}{\partial y} y - f
\end{equation}
which is, transforming the Lagrangian, equivalent to
\begin{equation}
  H(\vec q, \vec p, t) \,=\, \frac{\partial L}{\partial \dot{\vec q}}
\dot{\vec q} - L \,=\, p \dot{\vec q} - L ,
\end{equation}
where we introduced
\begin{equation}
  \frac{\partial L}{\partial \dot{\vec q}} \,=\, p
\end{equation}
as the \textit{canoncial momentum}. We know have transformed the Lagrangian via
the \textit{Legendre transformation} into the so calles \textit{Hamilton
function} $H(\vec q, \vec p, t)$.

In the next step we want to derive the \textit{Lagrange's equations} equivalent
the \textit{Hamilton's equations}, also called \textit{canonical equations}.
Hence regarding the total derivative of our newly introduced \textit{Hamilton
function}
\begin{equation}
  \begin{aligned}
    dH \,=\,& d(p \dot q) - dL \\
    \,=\,& \dot q dp + p d \dot q - \frac{\partial L}{\partial q} dq - \frac{d
L}{d \dot q} \dot q - \frac{\partial L}{t} dt \\
    \,=\,& \dot q dp + \frac{d L}{d \dot q} d \dot q - \frac{\partial L}{d q}
dq - \frac{\partial L}{d \dot q} d \dot q - \frac{\partial L}{dt} dt \\
    \,=\,& \dot q dp - \frac{L}{d q} dq - \frac{d L}{dt } dt
  \end{aligned}
\end{equation}
In addtion we know the total differential of $H(q, p, t)$ by
\begin{equation}
  d H(q, p, t) \,=\, \frac{\partial H}{d q} dq + \frac{\partial H}{\partial p}
dp + \frac{\partial H}{\partial t} dt.
\end{equation}
Thus by comparing the two differentials we get
\begin{equation}
  \label{canonicalEquation}
  \frac{\partial H}{\partial q} \,=\, - \frac{\partial L}{\partial q} \,=\,
\dot p, \quad \frac{\partial H}{\partial p} \,=\, \dot q \quad \text{and} \quad
\frac{\partial H}{\partial t} \,=\, \frac{\partial L}{\partial t}
\end{equation}
where we used the \textit{Lagrangian equation} to get
\begin{equation}
  \frac{d}{dt} \frac{\partial L}{\partial \dot q} - \frac{\partial L}{\partial
q} \,=\, \frac{d}{dt} p - \frac{\partial L}{\partial q} \quad \Rightarrow \quad
\dot p \,=\, \frac{\partial L}{\partial q}
\end{equation}
Eq.~(\ref{canonicalEquation}) is then referred to as \textit{Hamilton's} or
\textit{canonical Equations}

\section{Canonical Transformation}
The \textit{canonical transformation} is a \textit{phase transformation}
\begin{equation}
  (\vec q, \vec p) \,\to\, (\vec q', \vec p'),
\end{equation}
which conserves the \textit{canonical equations} 
\begin{equation}
  \dot{\vec q}' \,=\, \frac{\partial H'}{\partial \vec p'} \quad \text{and}
\quad \dot{\vec p}' \,=\, - \frac{\partial H'}{\partial \vec q'}.
\end{equation}
The transformed Hamilton function $H'$ proceeds from the transformation of H.
To perform this transformation we need to introduce \textit{Hamilton's modified
principle}
\begin{equation}
 \delta S[\vec q(t)] \,=\, \delta \int_{t_1}^{t_2} L(\vec q(t), \dot{\vec
q}(t), t) dt \,=\, 0 \quad \Rightarrow \quad \delta S [\vec q(t), \vec p(t)]
\,=\, \int_{t_1}^{t_2} \vec p \dot{\vec q} - H(\vec q(t), \vec p(t)) dt 
\end{equation}
One should notice that $\vec p$ and $\vec q$ are now independent but equally
valued variables. The \textit{path of motion} now takes place in the
\textit{phase space} $(\vec q, \vec p)$. 

Within \textit{Hamilton's principle} we can add the time derivative of an
arbitrarily function $dF/dt$, without changing the extremum of the action. We
can do so because the integrant $dF/dt$ only will be evaluated at the fixed
points $F(t_1)$ and $F(t_2)$, which will not be variated and consequently
vanish if F is a function of $(\vec q, \dot{\vec q}, t)$, $(\vec q, \vec p, t)$
or a mixture of the phase space coordinates. In addtion our transformed
variables have to satisfy \textit{Hamilton's principle}
\begin{equation} 
  \delta S[\vec q', \vec p'] \,=\, \delta \int_{t_1}^{t_2} \left[ \vec p' \dot{\vec
q}' - H'(\vec q', \vec p', t) + \frac{dF}{dt} \right] dt \,=\, 0.
\end{equation}
Compairing the two \textit{action integrals} yields
\begin{equation}
  \label{canonicalEquationCondition}
  \vec p \dot{\vec q} - H \,=\, \vec p' \dot{\vec q}' - H' + \frac{dF}{dt},
\end{equation}
where $F$ will be refered to as the \textit{generator function}. 

To show how the generator function specifies the equations of transformations,
suppose $F$ were given as 
\begin{equation}
  F \,=\, F_1(\vec q, \dot{\vec q}, t)
\end{equation}
Hence the total differential of $F_1$ is given by
\begin{equation}
  F_1 \,=\, \frac{\partial F_1}{\partial \vec q} dq + \frac{\partial F_1}{\partial
\vec q'} dq' + \frac{\partial F_1}{\partial t} dt
\end{equation}
Moreover we can also formulate the total differential of $F$ from rewriting
eq.~(\ref{canonicalTransformationCondition}) yielding
\begin{equation}
  \begin{aligned}
    dF_1 \,=\,& ( \vec p \dot{\vec q} - \vec p' \dot{\vec q}' + H' - H )dt \\
    \,=\,& \vec pdq - \vec p'dq + (H'-H)dt.
  \end{aligned}
\end{equation}
Compairing the two differentials gives us the transformation equations
\begin{equation}
  \label{transformationEquations}
  \vec p \,=\, \frac{F_1}{\partial \vec q} \, \quad \vec p' \,=\, -
\frac{\partial F_1}{\partial \vec q'} \quad \text{and} \quad H' \,=\, H +
\frac{\partial F_1}{\partial t}
\end{equation}

Remind yourself that we want to get $\vec q'$, $\vec p'$ and $H'$ to get the
\textit{equations of motions} from \textit{Hamilton's equations}. Thus if we
have $\vec q$, $\vec p$ and $F_1$ given we can use the former transformation
equations eq.~(\ref{transformationEquations}) to get our new variables in
dependency of our old ones
\begin{equation}
  \vec q'(\vec q, \vec p, t), \quad \vec p'(\vec q, \vec p, t) \quad
\text{and} \quad \vec H'(\vec q, \vec p, t)
\end{equation}
The first variable is given by
\begin{equation}
  \frac{\partial F_1(\vec q, \vec q', t)}{\partial \vec q} \,=\, \vec p(\vec q,
\vec q', t) \quad \longrightarrow \quad \vec q' (\vec q, \vec p, t),
\end{equation}
where we have rewritten $p(\vec q, \vec q', t) \to q'(\vec q, \vec p, t)$. To
get $\vec p'$ we use the second transformation equation
\begin{equation}
  \frac{\partial F_1(\vec q, \vec q', t)}{\partial \vec q'} \,=\, \vec p'(\vec
q, \vec q', t) \quad \longrightarrow \quad \vec p'(\vec q,\vec p, t),
\end{equation}
where we plugged $\vec q'$ into $\vec p'(\vec q, \vec q', t)$. Finally $H'$ is
given through the last transformation equation
\begin{equation}
  \frac{\partial F_1(\vec q, \vec q', t)}{\partial t} \,=\, H'(\vec q, \vec q',
t) \quad \longrightarrow \quad H'(\vec q, \vec p, t),
\end{equation}
where we plugged $\vec q'$ into $H'(\vec q, \vec q', t)$. Hence we are now left
with $\vec q'$, $\vec p'$ and $H'$ dependent on our given, old variables $\vec
q$, $\vec p$ and $H$.

There exist also other choises of generating functions, connected to $F_1$ via
the \textit{Legendre transformation}, which we will sum up in the follwoing
table.

\begin{center}
    \begin{tabular}{ | l | l | }
    \hline
    Generating  Function & Transformation functions \\ \hline
    $F \,=\, F_1(\vec q, \vec q', t)$ & $\vec p \,=\, \frac{\partial
F_1}{\partial \vec  q} \quad \vec P \,=\, - \frac{\partial F_1}{\partial \vec
q'} \quad H' \,=\, H + \frac{\partial F_1}{\partial t} $ \\
    $F \,=\, F_2(\vec q, \vec p', t) - \vec q' \vec p'$ & $\vec p \,=\,
\frac{\partial F_2}{\partial \vec q} \quad  \vec q' \,=\, \frac{\partial
F_2}{\partial \vec p'} \quad H' = H + \frac{\partial F_2}{\partial t}$ \\
    $F = F_3(\vec p, \vec q', t) + \vec q \vec p' $ & $\vec q \,=\,
-\frac{\partial F_3}{\partial \vec p} \quad \vec p' \,=\, - \frac{\partial
F_3}{\partial \vec q'} \quad H' = H + \frac{\partial
F_3}{\partial t}$ \\
    $F \,=\, F_4(\vec p, \vec p', t) + \vec q \vec p - \vec q' \vec p'$ & $\vec
q \,=\, - \frac{\partial F_4}{\partial \vec p} \quad \vec q' \,=\,
\frac{\partial F_4}{\partial \vec p'} \quad H' = H + \frac{\partial
F_4}{\partial t}$ \\
    \hline
    \end{tabular}
\end{center}





  


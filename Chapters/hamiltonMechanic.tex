\chapter{Hamilton Mechanic}
In this chapter we want to further develop our theory introducing the
Hamitlonian.

In the last two chapters we developed the \textit{Lagrange Mechanic}, which was
valid in all coordinates systems and did not include \textit{forces of
constraints}\footnote{In the classical Newton Mechanics, one could only use
cartesian coordinates and had to use difficult forces of constraint to solve
mechanical problems.}. The \textit{Lagrangian formalism} was developed from the
\textit{d'Alembert principle} or \textit{Hamilton principle}, which subsituted
the axioms of Newton. Restricting ourself to \textit{holonom conservativ}
problems we have to deal with 2S differential equations of 2nd order for $q_1,
\cdots, q_S$ generalized coordinates. Implementing the need of 2S initial
conditions for the solution. 
\section{Legendre Transformation}
\section{Hamilton's Equation of Motion}

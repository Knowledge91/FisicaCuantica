\chapter{Momentum Angular}

\begin{ejercicio}
\textbf{Full1: 15} Encontrad los valors propios de el operador $L_+L_-$.
Ecuentradlos tambien por el operador $L_-L_+$.
\end{ejercicio}
\begin{solucion}
Los operadores $L_+$ y $L_-$ estan dado por
$$
	L_+ = (L_x + iL_y) \quad \text{y} \qquad L_i = (L_x - iL_y)
$$ 
con el commutador y la relacion siguiente
$$
	[L_x, L_y] = i \hbar L_z \quad \text{y} \quad L^2 = L_x^2 + L_y^2 + L_z^2
$$
\red{explain quantization!}
Por eso podemos evaluar el operador $L_+L_-$
$$
	L_+L_- = (L_x + iL_y)(L_x - iL_y) = L_x^2 + L_y^2 \underbrace{+ iL_xL_y -
iL_y L_x}_{-i[L_x, L_y]} = L^2 - L_z^2 + \hbar L_z = \hbar^2 ((l+1) -m^2 +m)
$$
y el operador $L_-L_+$
$$
	L_-L_+ = (L_x - iL_y)(L_x + iL_y) = L_x^2 + L_y^2 + \underbrace{iL_xL_y -
iL_yL_x}_{i[L_x, L_y]} = L^2 - L_z^2 - \hbar L_z = \hbar^2 ((l+1) -m^2 -m)
$$
\end{solucion}
% -----------------------------------------------------------------------------

\begin{ejercicio}
\textbf{Full1: 18,19}
\begin{itemize}
\item 	Calculad $(\vec n \cdot \vec \sigma)^2 \equiv \sigma^2$, done $\vec n$ es un
vector unitario del espacio geometric y $\vec \sigma$ son los matrices de Pauli. 
\item	Con el resultado de la primera parte comprobad la igualidad
$e^{i\alpha(\vec n \cdot \vec \sigma} \equiv e^{i\alpha \sigma_{\vec n}} = \cos
\alpha + i \sigma_{\vec n} \sin \alpha$.
\end{itemize}
\end{ejercicio}
\begin{solucion}
\begin{itemize}
\item	
Por ese ejercicio conocemos las realciones de los matrices de Pauli que estan
dado por
$$
	{\sigma_i, \sigma_j} = \sigma_i \sigma_j - \sigma_j \sigma_i = 0 \quad
\text{y} \sigma_i^2 = \mathds{1}	
$$
Ahora podemos empazor con la calculacion del primer producto escalar
$$
	\sigma_{\vec n}^2 = (\vec n \vec \sigma)^2 = \sum_{i, j=0}^3 n_i n_j
\sigma_i \sigma_j = \sum_{i,j =1}^3 
$$
Considerando solo la suma de los matrices de Pauli nos da
$$
	\sigma_{i,j=1}^3 \sigma_i \sigma_j = \underbrace{\sigma_1 \sigma_1 + \sigma_2 \sigma_2
+ \sigma_3 \sigma_3}_{\mathds{1}} + \underbrace{\sigma_1 \sigma_2 +
\sigma_2\sigma_1}_0 + \underbrace{\sigma_1 \sigma_3 + \sigma_3 \sigma_1}_0 +
\underbrace{\sigma_2 \sigma_3 + \sigma_3 \sigma_2}_0
$$
Por lo tanto el resultado esta dado por
$$
	\vec \sigma_{\vec n}^2 = (\vec n \cdot \vec \sigma)^2 = \sum_{i,j = 1}^3 n_i
n_j \mathds{1} = \mathds{1} 
$$
Que ademas significa los importantes relaciónes por la segunda parte
$$
	\sigma_{\vec n}^{2n} = \mathds{1} \quad \text{y} \qquad \sigma_{\vec
n}^{2n+1} = \sigma_{n}
$$

\item Por la segunda parte desarollamos la función exponential y vemos que
podemos dividirla en un parte par y en un parte impar. El exponential
desarollado esta dado por
\begin{align*}
	e^{i\alpha \sigma_{\vec n}} &= 1 + i \alpha \sigma_{\vec n} - \frac{\alpha^2
\sigma_{\vec n}^2}{2!} + \frac{i \alpha^3 \sigma_{\vec n}^3}{3!} -
\frac{\alpha^4 \sigma_{\vec n}^4}{4!} + \cdots \\
	&= 1 + i \alpha \sigma_{\vec n} - \frac{\alpha^2
\mathds{1}}{2!} + \frac{i \alpha^3 \sigma_{\vec n}}{3!} -
\frac{\alpha^4 \mathds{1}}{4!} + \cdots \\
	&= \sum_{n=0}^\infty (-)^n \frac{\alpha^{2n+1} \sigma_{\vec
n}^{2n+1}}{(2n+1)!} + \sum_{n=0}^\infty (-)^n \frac{\alpha^{2n} \sigma_{\vec
n}^{2n}}{(2n)!} \\
	&= \sin{\alpha} + i \sigma_{\vec n} \sin(\alpha) \qed
\end{align*}
donde hemos utliziado la relacion del primer ejercicio por $\sigma_{\vec n}$ par
y impar y los desarollos del $\sin$ y $\cos$
\begin{align*}
	\sin(x) &= \sum_{n=0}^\infty \frac{(-)^n x^{2n+1}}{(2n +1)!} = x -
\frac{x^3}{3!} + \frac{x^5}{5!} - \frac{x^7}{7!} + \cdots \\
	\cos(x) &= \sum_{n=0}^\infty \frac{(-)^n x^{2n}}{(2n)!} = 1 -
\frac{x^2}{2} + \frac{x^4}{4!} - \frac{x^6}{6!} + \cdots
\end{align*}
\end{itemize}
\end{solucion}

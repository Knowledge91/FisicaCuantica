\chapter{Los postulados de la mecánica cuántica}

\section{Estados Puros}
\begin{definition}
	A la mecánica cuantica un estado es un vector $\psi \rangle$ (\textbf{vector
estado} o \textbf{ket}) normalizado ($\langle \psi | \psi \rangle = 1$) en un
espacio Hilbert $\mathcal{H}$ comlejo, completo, unitario y separable.
\end{definition}

\section{Observables}
\begin{definition}
Cada observable \textbf{A} de un systema físico se representa en la mecánica
cuantica mediante un operador \textbf{hermítico $\tilde A$}.
\end{definition}

\section{}


